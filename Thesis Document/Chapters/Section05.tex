% Chapter Template

% Main chapter title
%\chapter[toc version]{doc version}
\chapter{Results}

% Short version of the title for the header
%\chaptermark{version for header}

% Chapter Label
% For referencing this chapter elsewhere, use \ref{ChapterTemplate}
\label{Section6}

% Write text in here
% Use \subsection and \subsubsection to organize text
This research focused on European Portuguese as the target language to adapt new models and, as such, all our evaluation methods
were focused on methods used for European Portuguese. Past work has explored \unsure{add references to past work containing the datasets} different datasets
relevant for the Portuguese vocabulary. However, the majority of benchmakrs focus on the Brazilian Portuguese rather than European one.


\section{Evaluation Method}
For the target language of European Portuguese, we used \textit{CalamePT} and \textit{SuperGluePTPT}\unsure{Add references to both benchmarks} 
benchmarks which contained peer reviewed data in the target language.

\subsection{CalamePT}
This benchmark focused on text-completion with the right word.\\
It contains a list of 200 manually generated texts, where the last word can easily be deducted from the prior text.
An example exercise from this benchmark is "Ela correu durante horas para alcançar a linha de \textunderscore{chegada}" where "chegada" is the wanted
token.

So the model would receive everything BUT "chegada" as a prompt and if the first word generated was " chegada" it would receive a positive
score for that exercise.

This is then repeated throughout all of the 2076(200+1876)\unsure{Verify how many phrases the calamept dataset has} prompts.

\subsection{SuperGluePTPT}
This dataset was obtained by translating a dataset \unsure{Add reference to original English SuperGlue Dataset} onto European Portuguese,
and it was later peer reviewed \unsure{Add the perrcentage of data that was reviewed}.

The evaluation method focuses on asking questions of Yes/No and prompting the model to answer one or the other.
Some prompt engineering was needed to force the model to respond with only Yes/No and afterwards, we calculate the accuracy of answers.

Meaning if the model got correct 10 answers out of 100, it would get 10\% of score.


\section{}
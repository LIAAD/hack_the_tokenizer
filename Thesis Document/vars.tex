%-------------------------------------------------------------------------
% DOCUMENT VARIABLES
% NOTE: To remove links write \cmd{name} instead of \cmd[link]{name}
% NOTE: For aesthetics, at least one of these fields must be set: faculty, group
% or department
%-------------------------------------------------------------------------

% Your thesis title \ttitle
\thesistitle{Hack the Tokenizer}

% Your thesis type Doctoral Thesis or Masters Thesis \ttype
\thesistype{Masters Thesis}

% Your supervisor's name
\supervisor[mailto:example@fc.up.pt]{Alípio \textsc{Jorge}}

% Your supervisor's name
% To hide the Co-Supervisor field, just comment out \cosupervisor
\cosupervisor[mailto:name@host.com]{Hugo \textsc{Sousa}}

% Your degree name
\degree{MSc. Computer Science}

% Your name
\authors[mailto:up201704025@edu.fc.up.pt]{Luis \textsc{Pinto}}

% Your address. Can apparently be left blank.
\addresses{}

% Your subject area
\subject{Computer Sciences}

% Keywords for your thesis
\keywords{Computer Sciences, Machine Learning, Large Language Models, LLM, Tokenizer, Fertility, Natural Language Processing, NLP}

% Your university's name
\university{Universidade do Porto}
% Your university's name in capitals
\UNIVERSITY{UNIVERSIDADE DO PORTO}

% Your faculty's name
% To hide the Faculty field, just comment out \FACULTY and \Faculty
\faculty{Faculdade de Ciências da Universidade do Porto}
% Your faculty's name in capitals
\FACULTY{FACULDADE DE CIÊNCIAS DA UNIVERSIDADE DO PORTO}

% Your department's name
% To hide the Department field, just comment out \DEPARTMENT and \department
\department{Departamento de Ciências de Computadores}
% Your department's name in capitals
\DEPARTMENT{DEPARTAMENTO DE CIÊNCIAS DE COMPUTADORES}

% Your research group's name
% To hide the Group field, just comment out \GROUP and \group
%\group[http://www.groupurl.com]{Research Group}
% Your research group's name in capitals
%GROUP[http://www.groupurl.com]{RESEARCH GROUP}



% Research Questions
\newcommand{\RQone}{Can an English-trained model achieve comparable effectiveness in European Portuguese through strategic modifications of the tokenizer?}
\newcommand{\RQtwo}{What is the impact on model generation efficiency of tokenizer adaptation?}
\newcommand{\RQthree}{Does tokenizer adaptation affect all models equally, or are there differences based on model architecture and size?}
\newcommand{\RQfour}{Which embedding initialization strategies are most effective for integrating new tokens into a pre-trained model?}
\newcommand{\RQfive}{Does tokenizer adaptation reduce effectiveness in the model's source language?}